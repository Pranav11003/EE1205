\let\negmedspace\undefined
\let\negthickspace\undefined
\documentclass[journal,12pt,twocolumn]{IEEEtran}
\usepackage{cite}
\usepackage{amsmath,amssymb,amsfonts,amsthm}
\usepackage{algorithmic}
\usepackage{graphicx}
\usepackage{textcomp}
\usepackage{xcolor}
\usepackage{txfonts}
\usepackage{listings}
\usepackage{enumitem}
\usepackage{mathtools}
\usepackage{gensymb}
\usepackage{comment}
\usepackage[breaklinks=true]{hyperref}
\usepackage{tkz-euclide} 
\usepackage{listings}
\usepackage{gvv}                                        
\def\inputGnumericTable{}                                 
\usepackage[latin1]{inputenc}                                
\usepackage{color}                                            
\usepackage{array}                                            
\usepackage{longtable}                                       
\usepackage{calc}                                             
\usepackage{multirow}                                         
\usepackage{hhline}                                           
\usepackage{ifthen}                                           
\usepackage{lscape}
\usepackage{caption}
\newtheorem{theorem}{Theorem}[section]
\newtheorem{problem}{Problem}
\newtheorem{proposition}{Proposition}[section]
\newtheorem{lemma}{Lemma}[section]
\newtheorem{corollary}[theorem]{Corollary}
\newtheorem{example}{Example}[section]
\newtheorem{definition}[problem]{Definition}
\newcommand{\BEQA}{\begin{eqnarray}}
\newcommand{\EEQA}{\end{eqnarray}}
\newcommand{\define}{\stackrel{\triangle}{=}}
\theoremstyle{remark}
\newtheorem{rem}{Remark}
\begin{document}

\bibliographystyle{IEEEtran}
\vspace{3cm}

\title{10.5.2.14}
\author{EE23BTECH11003 - pranav}
\maketitle
\newpage

\bigskip
\renewcommand{\thefigure}{\arabic{figure}}
\renewcommand{\thetable}{\arabic{table}}

\textbf{Question}:Given that $\frac{dy}{dx}=2x+y$ and $y=1$,when $x=0$ Using Runge-Kutta fourth order method,the value of $y$ at $x=0.2$ is \hfill(GATE 2023 AG 50) 
\solution\\
By using runge kutta 4 th order method\\
\begin{table}[h]
    \centering
    \input{tables/Table.Tex}
    \caption{Variables Used}
    \label{ag:50}
\end{table}
\begin{align}
    y_{n+1}&= y_{n}+\frac{h}{6}(k_1+2k_2+2k_3+k_4)\\
    x_{n+1}&=x_{n}+h\\
    k_1&=f(x_n,y_n)\\
    k_2&=f(x_n+\frac{h}{2},y_n+h\frac{k_1}{2})\\
    k_3&=f(x_n+\frac{h}{2},y_n+h\frac{k_2}{2})\\
    k_4&=f(x_n+h,y_n+hk_3)
\end{align}
assume step size as $0.1$ and initial conditions as $x=0$ and $y=1$\\
\begin{align}
    k_1&=2(0)+1=1\\
    k_2&=2(0+\frac{0.1}{2})+(1+\frac{0.1}{2})\\
    \implies k_2&= 1.15\\
    k_3&=2(0+\frac{0.1}{2})+(1+\frac{0.115}{2})\\
    \implies k_3&=1.1575\\
    k_4&=2(0+0.1)+(1+0.11575)\\
    \implies k_4&=1.3158\\
    y_{n+1}&=1+\frac{0.1}{6}(1+2.30+2.315+1.3158)\\
    \implies y_{n+1}&=1.1155\\
    x_{n+1}&=0.1
\end{align}
cosidering outputs of last iteration as inputs of next iteration\\
\begin{align}
    k_1&=2(0.1)+1.1155=1.3155\\
    k_2&=2(0.1+\frac{0.1}{2})+(1.1155+\frac{0.12}{2})\\
    \implies k_2&= 1.4755\\
    k_3&=2(0.1+\frac{0.1}{2})+(1.1155+\frac{0.1475}{2})\\
    \implies k_3&=2.1532\\
    k_4&=2(0.1+0.1)+(1.1155+2.1532)\\
    \implies k_4&=3.6687\\
     y_{n+1}&=1.1155+\frac{0.1}{6}(1.3155+7.014+3.6687)\\
     \implies y_{n+1}&=1.319\\
     x_{n+1}&=0.2
\end{align}
so at $x=0.2$ value of $y$ is $1.319$
\end{document}
